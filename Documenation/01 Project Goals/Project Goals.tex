\documentclass[12pt, titlepage]{article}
\usepackage{tabularx}
\usepackage[left=2cm, right=2cm, top=2cm, bottom=2cm]{geometry}
\usepackage[utf8]{inputenc}
\usepackage{color}
\usepackage{graphicx}
\usepackage{amsmath}
\usepackage{tikz-timing}
\usepackage{setspace}
\doublespacing

\newcommand{\pnote}[1]{{\langle \text{#1} \rangle}}

\title{
McMaster University\\
SFWRENG 4G06\\
\bigskip\bigskip\bigskip
{\bf Tip of My Shoe - Project Goals\\}
\bigskip\bigskip\bigskip\bigskip
\begin{table}[h!]
\begin{center}
\begin{tabular}{|p{5cm}|p{5cm}|p{5cm}|}
	\hline
	\bf Member Name & \bf Mac ID & \bf Student Number\\
	\hline
	\hline
	Chris Dibussolo & dibussoc & 400070368\\
	\hline
  Andrew Lucentini & lucenta & 001430150\\
  \hline
	Owen McNeil & mcneilo & 400065750 \\
	\hline
	Daniel Scime & scimed1 & 400069926\\
	\hline
  Ashley Williams & willia18 & 400081787\\
	\hline
  Lucas Zacharewicz & zacharel & 400054446\\
	\hline
\end{tabular}
\end{center}
\end{table}
\date{October 7, 2019}
}

\usepackage{natbib}
\usepackage{graphicx}

\begin{document}

\maketitle

\newpage

% -----------------------------------------------------------------------------

\section{Project Description}

\paragraph{} Tip of My Shoe is a mobile application that aims to combine image recognition, machine learning and crowd sourcing in order to allow users to quickly identify the brand and model of a pair of shoes. Tip of My Shoe will operate in a manner similar to the popular song-identifying app \textbf{Shazam}, however instead of providing a sample of a song, the user will provide an image of the shoes that they'd like to identify. Upon successful matching, the application will attempt to provide approximate pricing, as well as links to where the user could purchase the footwear.

\section{Detailed Explanations Of Goals}

\paragraph{} Our first goal is to overcome the difficulties of image recognition and machine learning in order to provide customers an optimal product in comparison to existing, related products. A major obstacle to this goal will come from the variety of shapes and sizes a shoe could come in, as well as the condition a shoe could appear in an image when compared to a similar one in retail.

\paragraph{} The primary goal of the project is to create a program that can match images of shoes, specifically sneakers, to retail sneakers that are for sale in some area around the user, or online.

\paragraph{} A secondary goal to the above is to expand the program's capabilities to perform the same function for a wider range of shoes such as heels, boots and more.

\paragraph{} Another goal is to ensure we have sufficient privacy and security terms of conditions for future customers.

\paragraph{} In terms of user interfacing the goal is to provide a seamless interface that is easy to use and navigate, so as to not deter newer users that may be using the application simply out of curiosity.

\paragraph{} A secondary goal to interfacing is to give the user the option to use the program through a website on their computer as opposed to just their mobile device.
\end{document}